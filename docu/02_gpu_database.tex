\chapter{GPU Databases}
Related to the massive amount of data that is collected nowadays, the stagnation of CPU speed
and the trend to use the GPU for tasks like machine learning, the database developers have discovered the GPU to improve the performance of their products, too.
Hence the main idea of GPU databases is to perform some operations on the GPU for acceleration purposes.

\paragraph{Strengths}
GPUs do have their strengths on parallelable tasks.
This is due to the fact that GPUs can have thousands of cores and high bandwidth memory on each card.
Thus most of the GPU databases products focus on analytics.


\paragraph{Weaknesses} Beside of these strengths, GPU database host several pitfalls like transfer of the data from the CPU to the GPU,
 the memory limitations and the massiv costs of GPU servers.


\section{Components of a GPU database}
The current section will give you an overview of the components a GPU database consists of.
The paper \cite{bress2014gpu} of Sebastian Bress et al. does split the components into Functional and Non-Functional properties.
Since those categories are reasonable they are used in this paper as well.
The interesting Non-Functional properties of GPU database are performance and portability.
The next sub section will explain and list the Functional Properties.

\subsection{Functional properties}

\subsubsection{Storage system}
If we talk about storage systems there are several scenarios conceivable.
First we the Video Random Access Memory (VRAM) of the GPUs we have an additional storage medium next to the Random-Access Memory (RAM) and the hard disk.
The different mediums provides different advantages and disadvantages.
Hard disks are persistent, cheap and have a huge capacity.
Unfortunately they are pretty slow.
Random-Access Memory is very fast compared to hard disks but the transfer to the GPU is still a bottleneck and it isn't a persistent storage, either.
With Video Random Access Memory the bottleneck of the transfer disappears, but most of the time the storage capacity is highly limited.

\subsubsection{Storage model}
In terms of storage model, there are row stores or column stores \cite{abadi2008column}.
Row stores store table records in a sequence of rows.
Column stores store table records in a sequence of columns, the entries of a column is stored in contiguous memory locations.
The advantages of column stores are tasks like aggregations, though row stores are more efficient if the result of a query returns multiple rows.


\subsubsection{Processing model}
There are two processing models in modern databases tuple-at-a-time and operator-at-a-time.

\subsubsection{Query processing}

\subsubsection{Transaction support}

\newpage

\subsection{General}