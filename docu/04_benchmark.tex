\chapter{Benchmark}
This section does concerned with a benchmark between the two databases PostgreSQL 10.0 and MapD 3.3.1.
The dataset is based on a part of the New York City Taxi and Limousine Commission (TLC) Trip Record Data \cite{nyc}.
All queries were executed on the same server.

\section{Dataset}
\label{sec:dataset}
\begin{figure}[H]
\centering
\captionsetup{justification=centering}
\includegraphics[width=0.9\textwidth]{images/nyc_taxi.png}
\caption[Taxi Dropoffs]{TLC Trip Record Data}
\end{figure}

The New York City Taxi and Limousine Commission (TLC) Trip Record Data \cite{nyc} consists of trip records from the yellow and green taxis.
Furthermore there are data from the For-Hire Vehicle (FHV).
The data includes information about capturing pick-up and drop-off locations, times, trip distances, fares, rate types, and driver-reported passenger counts.

To simplify the handling with the huge amount of data we used the scripts from the Github repository \cite{nyctaxigithub}, that is able to download all the data,
provides the schema for PostgreSQL with the PostGIS \cite{postgis} extension and consists of scripts to import the data.

\subsection{Specification}
To get a brief overview of the specification have a look at the following list.
\begin{itemize}[noitemsep, topsep=0pt]
\itemsep-0.5em
 \item[Format:]  CSV
 \item[Yellow:]  January 2009 - June 2017
 \item[Green:]  August 2013 - June 2017
 \item[FHV:]  January 2015 - June 2017
 \item[Taxi trips:] Over 1.1 billion \cite{billion}
 \item[Size:] 267 GB \cite{billion}
\end{itemize}

\subsection{Data subset}
Due to the fact that not all GPU databases are able to handle queries based on data larger than the GPU memory,
we had to shrink the dataset fitting into the available memory.

The GPU used on the server for the benchmark is a Nvidia Tesla K40m with 12 GB memory.
Hence we used the yellow and green taxi trip data of the year 2015, that has a size of 12 GB, too.
Consequently all queries fit into the GPU memory.


\newpage
\section{Queries}
\label{sec:queries}
The Queries were predefined by Prof. Stefan Keller and are inspired by the Google BigQuery examples \cite{bigquery}.
All the queries are listed below, the used syntax is able to run on MapD. \\


\begin{lstlisting}[language=sql, caption={Query 1, Counts all the yellow taxi trips},captionpos=b]
SELECT passenger_count, Avg(total_amount)
FROM   trips
GROUP  BY 1;
\end{lstlisting}


\begin{lstlisting}[language=sql, caption={Query 2, Calculates the average passenger amount per trip},captionpos=b]
SELECT cab_type_id, Count(*)
FROM   trips
GROUP  BY 1;
\end{lstlisting}

\begin{lstlisting}[language=sql, caption={Query 3, Sums the yearly amount of passengers},captionpos=b]
SELECT passenger_count, Extract(year FROM pickup_datetime), Count(*)
FROM   trips
GROUP  BY 1, 2;
\end{lstlisting}


\begin{lstlisting}[language=sql, caption={Query 4, Groups the amount of passenger by year regarding the trip distance},captionpos=b]
SELECT passenger_count, Extract(year FROM pickup_datetime), Cast(trip_distance AS INT), Count(*)
FROM   trips
GROUP  BY 1, 2, 3
ORDER  BY 2, 4 DESC;
\end{lstlisting}

\begin{lstlisting}[language=sql, caption={Query 5, Queries all trips in a certain bounding box},captionpos=b]
SELECT *
FROM   trips
WHERE  ( pickup_longitude BETWEEN -74.007511 AND -73.983479 )
       AND ( pickup_latitude BETWEEN 40.7105 AND 40.731071 ) LIMIT 10;
\end{lstlisting}


\begin{lstlisting}[language=sql, caption={Query 6, Determines the average speed of the yellow taxi trips by hour of the day in a bounding box},captionpos=b]
SELECT Extract(HOUR FROM pickup_datetime) AS h, AVG(trip_distance / NULLIF(TIMESTAMPDIFF(HOUR,pickup_datetime, dropoff_datetime),0)) AS speed
FROM   trips
WHERE  ( pickup_longitude BETWEEN -74.007511 AND -73.983479 )
       AND ( pickup_latitude BETWEEN 40.7105 AND 40.731071 )
       AND trip_distance > 0
       AND fare_amount / trip_distance BETWEEN 2 AND 10
       AND dropoff_datetime > pickup_datetime
       AND cab_type_id = 1
GROUP  BY h
ORDER  BY h;
\end{lstlisting}


\begin{lstlisting}[language=sql, caption={Query 7, Computes the average speed of the yellow taxi trips by hour of the day},captionpos=b]
SELECT Extract(HOUR FROM pickup_datetime) AS h, Avg(trip_distance / NULLIF(TIMESTAMPDIFF(HOUR, pickup_datetime, dropoff_datetime),0))AS speed
FROM   trips
WHERE  trip_distance > 0
       AND fare_amount / trip_distance BETWEEN 2 AND 10
       AND dropoff_datetime > pickup_datetime
       AND cab_type_id = 1
GROUP  BY h
ORDER  BY h;
\end{lstlisting}


\begin{lstlisting}[language=sql, caption={Query 8, Calculates the average speed of the yellow taxi trips by day of the week},captionpos=b]
SELECT Extract(DOW FROM pickup_datetime) AS dow, Avg(trip_distance / NULLIF(TIMESTAMPDIFF(HOUR,pickup_datetime,  dropoff_datetime), 0)) AS speed
FROM   trips
WHERE  trip_distance > 0
       AND fare_amount / trip_distance BETWEEN 2 AND 10
       AND dropoff_datetime > pickup_datetime
       AND cab_type_id = 1
GROUP  BY dow
ORDER  BY dow;
\end{lstlisting}


\begin{lstlisting}[language=sql, caption={Query 9, Determines the average speed of the yellow taxi trips by day of the week in a bounding box},captionpos=b]
SELECT Extract(DOW FROM pickup_datetime) AS dow, Avg(trip_distance / NULLIF(TIMESTAMPDIFF(HOUR,pickup_datetime,  dropoff_datetime), 0)) AS speed
FROM   trips
WHERE  ( pickup_longitude BETWEEN -74.007511 AND -73.983479 )
       AND ( pickup_latitude BETWEEN 40.7105 AND 40.731071 )
       AND trip_distance > 0
       AND fare_amount / trip_distance BETWEEN 2 AND 10
       AND dropoff_datetime > pickup_datetime
       AND cab_type_id = 1
GROUP  BY dow
ORDER  BY dow;
\end{lstlisting}

\newpage
\section{Results}
As already mentioned the queries of the benchmark were applied on a database without GPU acceleration,
in our case PostgreSQL 10.0 with the PostGIS 2.4 extension and a on MapD 3.3.1 a GPU powered database.
The dataset was introduced in section~\ref{sec:dataset} and the queries are listed in the section~\ref{sec:queries}.

\subsection{PostgreSQL vs. MapD cold}
The diagram \ref{fig:cold_postgres_vs_mapd} compares the query duration of PostgreSQL and MapD at the first time of execution (cold).
That means there might are no caching effects influencing the measurement.

\begin{minipage}{\textwidth}
  \begin{minipage}[b]{0.59\textwidth}
    \centering
     \includegraphics[width=1.0\textwidth,height=1.0\textheight,keepaspectratio]{images/cold_postgres_vs_mapd.png}
    \captionof{figure}{Diagram PostgreSQL vs. MapD cold \\ (logarithmic scale)}
    \label{fig:cold_postgres_vs_mapd}
  \end{minipage}
  \hfill
  \begin{minipage}[b]{0.39\textwidth}
    \centering
  \begin{tabular}{ |p{1cm}|p{2cm}|p{2cm}| }
    \hline
    Query & PostgreSQL [ms] & MapD [ms] \\
    \hline
    1 & 64878 & 203 \\
    2 & 74404 & 456 \\
    3 & 145749 & 742 \\
    4 & 172592 & 1431 \\
    5 & 2556 & 663 \\
    6 & 89029 & 5756 \\
    7 & 150823 & 119 \\
    8 & 152883 & 315 \\
    9 & 87535 & 243 \\
    \hline
\end{tabular}
      \captionof{table}{Data PostgreSQL vs. MapD cold}
            \label{tab:cold_postgres_vs_mapd}
    \end{minipage}
 \end{minipage}

\newpage
\subsection{PostgreSQL vs. MapD warm}
The diagram \ref{fig:warm_postgres_vs_mapd} compares the query duration of PostgreSQL and MapD after the first time of execution (warm).
That means there might are caching effects speeding up the measurement.
The query was processed 5 times and the values listed in the table \ref{tab:warm_postgres_vs_mapd} are the mean time of these measurements.

\begin{minipage}{\textwidth}
  \begin{minipage}[b]{0.59\textwidth}
    \centering
     \includegraphics[width=1.0\textwidth,height=1.0\textheight,keepaspectratio]{images/warm_postgres_vs_mapd.png}
    \captionof{figure}{Diagram PostgreSQL vs. MapD warm \\ (logarithmic scale)}
    \label{fig:warm_postgres_vs_mapd}
  \end{minipage}
  \hfill
  \begin{minipage}[b]{0.39\textwidth}
    \centering
  \begin{tabular}{ |p{1cm}|p{2cm}|p{2cm}| }
    \hline
    Query & PostgreSQL [ms] & MapD [ms] \\
    \hline
    1 & 62684 & 41 \\
    2 & 75140 & 223 \\
    3 & 142338 & 412 \\
    4 & 172052 & 840 \\
    5 & 1685 & 25 \\
    6 & 84439 & 149 \\
    7 & 150073 & 86 \\
    8 & 155107 & 115 \\
    9 & 86060 & 131 \\
    \hline
\end{tabular}
      \captionof{table}{Data PostgreSQL vs. MapD warm}
            \label{tab:warm_postgres_vs_mapd}
    \end{minipage}
 \end{minipage}



\subsection{PostgreSQL cold vs. warm}
The diagram \ref{fig:postgres_cold_vs_warm} compares the query duration of PostgreSQL at the first execution with the mean of the next 5 executions.

 \begin{minipage}{\textwidth}
  \begin{minipage}[b]{0.59\textwidth}
    \centering
     \includegraphics[width=1.0\textwidth,height=1.0\textheight,keepaspectratio]{images/postgres_cold_vs_warm.png}
    \captionof{figure}{Diagram PostgreSQL cold vs. PostgreSQL warm}
    \label{fig:postgres_cold_vs_warm}
  \end{minipage}
  \hfill
  \begin{minipage}[b]{0.39\textwidth}
    \centering
  \begin{tabular}{ |p{1cm}|p{2cm}|p{2cm}| }
    \hline
    Query & PostgreSQL cold [ms] & PostgreSQL warm [ms] \\
    \hline
    1 & 64878 & 62684 \\
    2 & 74404 & 75140 \\
    3 & 145749 & 142338 \\
    4 & 172592 & 172052 \\
    5 & 2556 & 1685 \\
    6 & 89029 & 84439 \\
    7 & 150823 & 150073 \\
    8 & 152883 & 155107 \\
    9 & 87535 & 86060 \\
    \hline
\end{tabular}
      \captionof{table}{Data PostgreSQL cold vs. PostgreSQL warm}
            \label{tab:postgres_cold_vs_warm}
    \end{minipage}
 \end{minipage}


\newpage
\subsection{MapD cold vs. warm}
The diagram \ref{fig:mapd_cold_vs_warm} compares the query duration of MapD at the first execution with the mean of the next 5 executions.


  \begin{minipage}{\textwidth}
  \begin{minipage}[b]{0.59\textwidth}
    \centering
     \includegraphics[width=1.0\textwidth,height=1.0\textheight,keepaspectratio]{images/mapd_cold_vs_warm.png}
    \captionof{figure}{Diagram MapD cold vs. MapD warm}
    \label{fig:mapd_cold_vs_warm}
  \end{minipage}
  \hfill
  \begin{minipage}[b]{0.39\textwidth}
    \centering
  \begin{tabular}{ |p{1cm}|p{2cm}|p{2cm}| }
    \hline
    Query & MapD cold [ms] & MapD warm [ms]\\
    \hline
    1 & 203 & 41 \\
    2 & 456 & 223 \\
    3 & 742 & 412 \\
    4 & 1431 & 840 \\
    5 & 663 & 25 \\
    6 & 5756 & 149 \\
    7 & 119 & 86 \\
    8 & 315 & 115 \\
    9 & 243 & 131 \\
    \hline
\end{tabular}
      \captionof{table}{Data MapD cold vs. MapD warm}
            \label{tab:mapd_cold_vs_warm}
    \end{minipage}
 \end{minipage}


\section{Conclusion}