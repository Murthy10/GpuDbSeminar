\definecolor{bgcolor}{rgb}{1, 1, 1}
\definecolor{codecolor}{rgb}{0, 0, 0}
\definecolor{stringcolor}{rgb}{0.62, 0, 0.36}
\definecolor{keywordcolor}{rgb}{0, 0, 0}
\definecolor{commentcolor}{rgb}{0.66, 0.66, 0.61}
\definecolor{linenumbercolor}{rgb}{0.4, 0.4, 0.4}
\definecolor{directivecolor}{rgb}{0, 0, 0}
\newfontfamily\listingsfont[Scale=.8]{Roboto}

\lstdefinelanguage{CustomAndroid}[]{C++} {
	morekeywords    = {*, func, var, let, nil, guard, protocol, extension, import, in, inout, typealias, associatedtype, precedencegroup},
	directivestyle  = \listingsfont\bfseries\color{directivecolor},
	morecomment=[n]{/*}{*/},    % allows for nested multi-line comments
}

\lstdefinestyle{Basic}{
	xleftmargin=15pt,
	backgroundcolor         = \color{bgcolor},
	fillcolor               = \color{bgcolor},
	basicstyle              = \linespread{1.0}\color{codecolor}\scriptsize\listingsfont,
	commentstyle            = \color{commentcolor},                  
	keywordstyle            = \listingsfont\bfseries\color{keywordcolor},
	identifierstyle         = ,
	numberstyle             = \color{linenumbercolor},                        % the style that is used for the line-numbers
	stringstyle             = \color{stringcolor},                            % string literal style
	breakatwhitespace       = false,                                          % sets if automatic breaks should only happen at whitespace
	breaklines              = true,                                           % sets automatic line breaking
	escapeinside            = {\%*}{*)},                                      % if you want to add LaTeX within your code
	extendedchars           = true,                                           % lets you use non-ASCII characters; for 8-bits encodings only
	rulesep                 = 5pt,
	framexleftmargin        = 2pt,
	framerule               = 0.5pt,
	framesep                = 1pt,
	keepspaces              = true,                                           % keeps spaces in text, useful for keeping indentation of code
	morekeywords            = {*,interface},      	 	  % if you want to add more keywords to the set
	numbers                 = left,                               % where to put the line-numbers; possible values are (none, left, right)
	numbersep               = 6pt,                                % how far the line-numbers are from the code
	showspaces              = false,                              % show spaces everywhere adding particular underscores
	showstringspaces        = false,                              % underline spaces within strings only
	showtabs                = false,                              % show tabs within strings adding particular underscores
	stepnumber              = 1,                                  % the step between two line-numbers. If it's 1, each line will be numbered
	tabsize                 = 2,                                  % sets default tabsize to 2 spaces
	title                   = \lstname,                           % show the filename of files included with \lstinputlisting
}

